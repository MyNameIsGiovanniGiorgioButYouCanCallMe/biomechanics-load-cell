Das Projekt zeigte, dass asymmetrische Belastungen bei Kniebeugen häufig vorkommen und sowohl durch Fehlhaltungen als auch durch muskuläre Dysbalancen bedingt sein können.
Die erste Messung zeigte bei allen Probanden eine ungleichmäßige Muskelbelastung, was eine erhöhte Belastung auf einer Seite zur Folge haben kann.
Nach dem regelmäßigen Training mit spezifischen Ausgleichsübungen konnte eine deutliche Verbesserung des Symmetrie-Indexes festgestellt werden.
Das verdeutlicht die Bedeutung von gezielten Korrekturübungen zur Prävention von Verletzungen und zur Optimierung von Bewegungsabläufen.
\\
\\
Obwohl nicht alle Dysbalancen komplett beseitigt werden konnten, zeigt das Projekt dennoch, dass gezielte Übungen einen spürbaren Einfluss auf die Symmetrie der Muskelbelastung haben.
Unterschiede in der Anatomie oder neurologische Faktoren könnten eine Rolle spielen, warum einige Asymmetrien bestehen bleiben.
Ein Verbesserungsvorschlag für zukünftige Versuche wäre, die Wägezellen sowohl vor als auch nach den Ausgleichsübungen einzusetzen, um eine vorher-nachher-Analyse anhand vergleichbarer Messdaten zu ermöglichen.
Die Reduktion der Asymmetrie konnte daher primär durch die EMG-Daten nachgewiesen werden.
Auch interessant wäre die Analyse einer einzelnen Kniebeuge mit den Wägezellen gewesen,  um die Bewegungsabläufe gezielter mit den Messwerten abzugleichen und asymmetrische Belastungen in spezifischen Phasen der Übung besser zu erkennen.
\\
\\
Für zukünftige Untersuchungen wäre es spannend, weitere Messungen durchzuführen, um den langfristigen Effekt der Übungen zu überprüfen und ob die Verbesserungen von dauer sind oder ab wann diese abklingen.
Auch interessant wäre die wöchentliche Verbesserung durch die Ausgleichsübungen zu analysieren, ob diese linear verläuft oder sich sprunghaft verbessert.
\\
Obwohl die Stichprobe mit nur vier Probanden sehr klein ist und die Ergebnisse daher nicht statistisch aussagekräftig sind, ist dennoch bemerkenswert, dass eine klare Verbesserung des Symmetrie-Indexes festgestellt werden konnte.
Dies zeigt, dass selbst mit kleinen Stichprobe interessante Einblicke gewonnen werden können.
\\
\\
Zusammenfassend zeigt das Projekt, dass asymmetrische Muskelbelastungen ein verbreitetes Phänomen sind, das durch gezielte Übungen reduziert werden kann. Dies hat sowohl für Freizeitsportler als auch für die physiotherapeutische Praxis Relevanz.