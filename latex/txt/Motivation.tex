
Ambidextrie beschreibt die Fähigkeit mit beiden Händen gleich geschickt zu sein \cite{wissen.de}. Da die meisten  Menschen nicht ambidextr sind, haben die sie eine stärkere und eine schwächere Seite. Das ist beim Schreiben oder Malen nicht weiter problematisch. Werden aber bei Belastung wie körperlicher Arbeit oder beim Sport Muskeln unterschiedlich stark belastet, kann das zu Fehlhaltungen und Schmerzen führen. Dadurch können Gelenke ungünstig belastet werden, was zu Gelenkverschleiß führen kann.\\
Muskuläre Dysbalancen können verschiedene Ursachen haben. \\
\\
 \subsection{Fehlhaltung im Alltag}
 Personen die im Alltag überwiegend sitzen nehmen häufig eine Fehlhaltung ein, was zu einer Dysbalance zwischen Brust- und oberer Rückenmuskulatur führen kann. Auch werden viele anstrengende Aufgaben, wie das Tragen von schweren Taschen oder das Öffnen von Marmeladengläsern, meist mit der stärkeren Hand erledigt. Dadurch wird diese Seite sträker und beweglicher, was zu Schulterproblemen führen kann.

 \subsection{Fehlhaltung nach Verletzungen}
 Durch Verletzungen können kompensierende Fehlhaltungen entstehen. Ein verletztes Knie kann zum Beispiel dazu führen, dass Menschen mit der verletzten Seite weniger stark auftreten. Das unverletzte Bein wird dadurch stärker beansprucht, wodurch das gesunde Bein stärker beansprucht wird und die Muskeln am verletzten Knie sogar weiter abnehmen.

 \subsection{Falsches Training}
Werden beim Kraftsport Übungen fehlerhaft ausgeführt, gleicht der Körper das ungleiche Kräfteverhältnis meist unbemerkt aus.

\begin{quote}
Steht das Becken nach rechts schief, dreht der Brustwirbel den Oberkörper nach links ein, das Kopfgelenk neigt den Kopf wieder nach rechts und so weiter – bei Sportlern findet meistens eine Art Kettenreaktion von unten nach oben statt. Dann sind die Augen wieder horizontal und man merkt nicht, dass man überhaupt eine Fehlhaltung eingenommen hat" \cite{MensHealth}
\end{quote}
Bei Kniebeugen kann es dann zum Beispiel passieren, dass sich der Sportler auf eine Seite lehnt und dadurch das kräftigere Bein stärker beansprucht als das ohnehin bereits schwächere Bein. \\
\\

