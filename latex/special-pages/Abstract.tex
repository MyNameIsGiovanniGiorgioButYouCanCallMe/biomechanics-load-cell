Dieses Projekt untersucht asymmetrische Muskelbelastungen beim Ausführen von Kniebeugen vor und nach spezifischen Ausgleichsübungen. 
Ziel war es, potenzielle Dysbalancen zwischen der linken und rechten Körperseite zu identifizieren und durch gezielte Übungen zu reduzieren. 
Die Messungen erfolgten sowohl mit Wägezellen als auch mittels Elektromyographie (EMG). 
Dabei zeigte sich, dass bei allen Probanden Asymmetrien in der Gewichtsverteilung und Muskelaktivierung vorlagen. 
Nach einem vierwöchigen Training mit Ausgleichsübungen konnte eine Verbesserung der Symmetrie festgestellt werden. 
Die Ergebnisse verdeutlichen, dass gezielte Ausgleichsübungen effektiv Dysbalancen reduzieren können, was sowohl im Alltag als auch im sportlichen Bereich von Bedeutung ist.